\documentclass[a4paper,
               10pt,
               fleqn]{article}

\author{Ervin Mazlagic}
\title{VIM-R}

\usepackage{luxpaper}
\usepackage{luxtitle}
 
\begin{document}
\luxtitle{Papers}
         {VIM-R}
         {Ervin Mazlagi\'c}
         {Adligenswil}
         {2013}

\tableofcontents
\newpage

\section{Was ist ``VIM-R''?}
Als \emph{VIM-R} wird das ganze Setup bezeichnet, mit welchem man R innerhalb
von VIM nutzen kann und dies so, wie man es von IDEs gewohnt ist.

Mit der Konfiguration die vorgestellt wird, kann innerhalb von VIM R-Code
geschrieben werden und in einer R-Konsole innerhalb von VIM ausgeführt werden.
Der Benutzer muss sich also nicht mit Ressourcenlähmenden Java-IDEs
herumschalgen oder auf den Konfort seines Lieblingseditors verzeichten.

\section{Installation}
Die hier vorgestellte Installation und Konfiguration bezieht sich auf 
Arch-Linux und kann nahezu identisch für andere Linux-Distributionen 
angewendet werden.

\subsection{Vor der Installation}
Es empfiehlt sich ein Backup des VIM-Verzeichnis (also \verb!~/.vim!) zu 
erstellen, da es überschrieben werden kann durch die Installation!

\subsection{R}
Zunächst muss R installiert werden
\begin{lstlisting}
pacman -S r
\end{lstlisting}
Da R einige Abhängigkeiten hat und spezielle Pakete noch in Fortran 
kompiliert werden müssen, ist es empfehlenswert die folgenden Pakete
ebenfalls zu installieren.
\begin{lstlisting}
pacman -S tcl tk gcc-fortran
\end{lstlisting}
Nun kann R im Terminal bereits ausgeführt werden mit dem Kommando
\verb!R!. Die R-Konsole kann offen gelassen werden für die nächsten 
Schritte der Installation.

\subsection{R-Zusatzpakete}
Für die Benutzung von R mittels VIM werden weitere Pakete benötigt wie etwa
\verb!vimcom! und \verb!setwidth!.
Diese können aber direkt über die R-Konsole installiert werden.
In der R-Konsole tippt man hierzu folgendes ein
\begin{lstlisting}
install.packages("vimcom", dependencies=TRUE)
install.packages("setwidth", dependencies=TRUE)
\end{lstlisting}
Es kann sein, dass die Installation von Paketen erfordert, dass R von root
bzw. mit sudo gestartet wurde.
Weiter ist es sinnvoll die folgenden Zeilen zum \verb!~/.Rprofile! zu ergänzen.
\begin{lstlisting}
if(interactive()){
        library(colorout)
	library(setwidth)
	library(vimcom)
}
\end{lstlisting}

\subsection{TMUX}
Damit das ganze Funktioniert, muss \verb!tmux! installiert sein, denn es
``klebt'' mehrere Terminals oder Konsolen an- bzw. übereinander.
\begin{lstlisting}
pacman -S tmux
\end{lstlisting}

\subsection{Screen-Plugin}
Bevor das eigentliche VIM-R-Plugin installiert werden kann, muss vorher noch
ein anderes Plugin zu VIM hinzugefügt werden namens \verb!screen!. 
Hierzu lädt man sich den sog. \verb!vimball! von screen herunter.
Danach öffnet man dieses mit VIM und macht daraus ein VIM-Source indem man
folgendes in VIM eingibt.
\begin{lstlisting}
:so %
\end{lstlisting}

\subsection{vimrc}
Damit man keine Probleme hat bei der Benutzung sollte VIM vorher noch 
passend konfiguriert werden. Die folgenden Zeilen sollten in der
\verb!vimrc! nicht fehlen.
\begin{lstlisting}
set nocompatible
syntax enable
filetype plugin on
filetype indent on
\end{lstlisting}

\subsection{R-Plugin}
Nun folgt der Kernteil der Installation, nämlich das eigentliche R-Plugin
für VIM. 
Zunächst lädt man sich das Plugin herunter und entpackt es in das 
VIM-Verzeichnis.
\begin{lstlisting}
unzip vim-r-plugin-0.9.9.2.zip -d ~/.vim
\end{lstlisting}
Danach macht man ein entsprechendes \verb!helptag! für das Plugin.
Hierzu öffnet man VIM und gibt das folgende ein.

\begin{lstlisting}
:helptags ~/.vim/doc
\end{lstlisting}
An diesem Punkt ist die Installation fertig. Zusätzlich können noch einige
Optimierungen für TMUX vorgenommen werden. Diese erlauben z.B. den Wechsel
zwischen den PANES per Mausklick (dies sollte eigentlich alles schon durch
das r-plugin vorgenommen worden sein). Hierzu kann die \verb!.tmux.conf!
erweitert werden mit den folgenden Zeilen.

\begin{lstlisting}
set-option -g prefix C-a
unbind-key C-b
bind-key C-a send-prefix
set-window-option -g mode-keys vi
set -g terminal-overrides 'xterm*:smcup@:rmcup@'
set -g mode-mouse on
set -g mouse-select-pane on
set -g mouse-resize-pane on
\end{lstlisting}

\section{Benutzung}

\subsection{Einfach}
Um nun R mit VIM zu coden kann in einen Terminal \verb!tmux! ausgeführt
werden. Danach öffnet man ein R-Skript (die Datei muss entweder die Endung
\verb!.R! oder \verb!.Rnw! haben) mit VIM. Bis zu diesem Punkt ist für den
Benutzer kein Unterschied vorhanden. Jetzt aktiviert man das Plugin mit
\begin{lstlisting}
\rf
\end{lstlisting}
Dies spaltet nun den VIM-Screen und es kommt eine R-Konsole hervor. Falls
die Keybindings von tmux richtig gesetzt sind kann man nun per Mausklick
vom VIM-Teil in den R-Teil wechseln. Falls die Keybindings nicht funktionieren
kann dies mittels \verb!<Ctrl-b>o! erfolgen.

Die Bedienung ist sehr vielfältig und kann online nachgelesen werden.
Der wichtigste Befehl ist \verb!\l!, denn damit sendet man eine Zeile
aus dem VIM-Teil ind den R-Teil und führ diese auch gleich aus.
Nebst einer Vielzahl von Kommandos sind auch viele Shortcuts für das
Skripten an sich vorhanden. Beispielsweise kann mit \verb!Ctrl+_!
das die oft benutzte Zuseisung \verb!<-! gesetzt werden (im VIM-Teil).

\subsection{Erweitert}
Das Set ``TMUX + VIM + R'' kann auch etwas einfacher verwendet bzw.
gestartet werden, indem man eine bestehende Sitzung in dieser 
Konfiguration ``detached'' und zu einem beliebigen Zeitpunkt dann
wieder ``reattached''. 

Zuerst startet man die TMUX Sitzung wie gewohnt indem man im Terminal
\verb!tmux! eingibt. Dann öffnet man eine R-Skript z.B. 
\verb!vim mystatistic.R! und startet die R-Konsole in VIM mit
\verb!\rf!. Nun kann man normal skripten und zu einem beliebigen 
Zeitpunkt die ganze Sitzung in den Hintergrund verschwinden lassen
mittels des Keybindings \verb!<Ctrl-b>d!. Nun ist man wieder in seinem
Terminal und sieht die Meldung, dass TMUX \verb![detached]! wurde.
Jetzt kann man zu einem gewünschten Zeitpunkt einfach mit der Anweisung
\verb!tmux attach! die alte Sitzung öffnen ohne die ganze Startprozedur
durchmachen zu müssen.

\section{Probleme}
Bei der Installation auf Arch funktionieren die Keybindings nicht,
z.B. kann die Maus nicht verwendet werden um zwischen den panes zu 
switchen (entsprechende Anweisung ist in der \verb!.tmux.conf!
eingetragen).

\textbf{Reboot löst das Problem}

\section{Weitere Informationen}
ArchWiki --- R\\
\indent \url{https://wiki.archlinux.org/index.php/R}\\
ArchWiki --- Tmux\\
\indent \url{https://wiki.archlinux.org/index.php/Tmux#Installation}\\
vim.org --- Vim-R-plugin\\
\indent \url{http://www.vim.org/scripts/script.php?script_id=2628}


\end{document}
